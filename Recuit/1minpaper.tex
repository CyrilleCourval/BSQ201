\documentclass[12pt]{article}

\usepackage{natbib}
\usepackage[utf8]{inputenc}
\usepackage{latexsym,amsfonts,amssymb,amsthm,amsmath}
\usepackage{afterpage}

\setlength{\parindent}{0in}
\setlength{\oddsidemargin}{0in}
\setlength{\textwidth}{6.5in}
\setlength{\textheight}{8.8in}
\setlength{\topmargin}{0in}

\usepackage[french,english]{babel}
\usepackage[T1]{fontenc}
\usepackage{graphicx}
\usepackage{subcaption}
\usepackage{float}
\usepackage{placeins}
\usepackage{url}

\title{Applications du calcul analogique}
\author{Cyrille Courval}
\date{\selectlanguage{french}\today}%

\begin{document}

\maketitle
\thispagestyle{empty}

\paragraph{\normalfont Le calcul analogique quantique, contrairement au calcul digital qui manipule des bits discrets, exploite l’évolution naturelle d’un système physique vers son état fondamental. Cette approche est mise en œuvre notamment à travers le recuit quantique.
Le recuit quantique permet de résoudre des problèmes NP-difficiles et d’optimisation en exploitant le phénomène de l’effet tunnel, ce qui facilite la convergence vers un minimum global. La résolution s’effectue à l’aide d’un Hamiltonien dépendant du temps, progressivement transformé pour guider le système vers la solution optimale.}

\paragraph{\normalfont Le calcul analogique peut se révéler fort intéressant dans certains domaines où le calcul digital atteint rapidement ses limites. Par exemple, dans le domaine de l'optimisation combinatoire, les problèmes complexes comme le fameux problème du voyageur de commerce, qui sont difficiles à résoudre classiquement, peuvent être 
abordés de manière plus rapide et efficace avec des techniques basées sur le recuit quantique. Une autre application potentielle serait en simulation moléculaire pour améliorer la découverte de nouveaux médicaments. En effet, le recuit quantique aurait le potentiel d'accélérer le processus d'évaluation de différentes combinaisons moléculaires, qui coûte beaucoup de temps et d'argent, en explorant efficacement l'espace des solutions pour des 
problèmes d'optimiation moléculaire.}

\paragraph{\normalfont Similairement, la recherche en alliages à haute entropie pourrait bénéficier du calcul analogique par recuit quantique. En modélisant les différentes structures atomiques et moléculaires possibles avec fiabilité, cette approche permettrait de prédire et construire de nouveaux matériaux avec des propriétés spécifiques. 
Cette application me semble particulièrement prometteuse, car le champ d'utilisation de ces alliages à haute entropie est vaste et la découverte de nouveaux matériaux pourrait avoir un impact significatif sur de nombreux secteurs. Les alliages à haute entropie sont déjà reconnus pour leur robustesse, leur stabilité thermique et leur résistance à la corrosion, ce qui les rend particulièrement adaptés à des secteurs exigeants tels que l'aérospatial 
et l'énergie nucléaire, où la durabilité des matériaux est cruciale. Ces alliages présentent également un fort potentiel dans le développement de matériaux magnétiques doux, qui sont utilisés notamment dans les moteurs électriques et les générateurs, car ils pourraient améliorer leur durée de vie et leur efficacité énergétique. Bref, en changeant les matériaux, on change ce qui est possible, et le recuit quantique ouvre des portes à la découverte et 
à la conception de matériaux sur mesure pour des applications industrielles et technologiques avancées. }

\end{document}